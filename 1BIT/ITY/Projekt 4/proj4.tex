%Ctvrty projekt do predmetu ITY 2013/2014
%Autor: Lukas Pelanek
%email: xpelan03@stud.fit.vutbr.cz


\documentclass[11pt, a4paper]{article}

\usepackage{times}
\usepackage[T1]{fontenc}
\usepackage[left=2cm, text={17cm, 24cm}, top=3cm]{geometry}
\usepackage[czech]{babel}
\usepackage[utf8]{inputenc}

\newcommand{\myuv}[1]{\quotedblbase #1\textquotedblleft}

\begin{document}
\begin{titlepage}

\begin{center}
	\Huge
 	\textsc{Vysoké učení technické v~Brně\\
 		{\huge Fakulta informačních technologií}}\\
 		
	\vspace{\stretch{0.382}}
	\LARGE
	 Typografie a publikování - 4. projekt\\ 
	 	\textbf{{\huge Bibliografické citace}}
	\vspace{\stretch{0.618}}
\end{center}

{\Large \today \hfill Lukáš Pelánek}
\end{titlepage}


\section*{Úvod} 
Pod pojmem latex si většina lidí, kteří se o~tuto oblast nezajímají, představí něco zcela odlišeného. Nicméně mám na mysli profesionální typografický nástroj {\LaTeX} [latech]. Pomocí {\LaTeX}u můžeme vytvářet dokumenty ve velmi vysoké typografické kvalitě. Ačkoliv práce s~{\LaTeX}em je zprvu složitá a náročná, po určitém čase, když si práci s~tímto nástrojem osvojíte, budete mistři v~psaní dokumentů.

\section*{Začínáme s~{\LaTeX}em}
Práce s~{\LaTeX}em se od většiny typografických nástrojů liší především v~tom, že vše sázíte pomocí příkazů. Dalo by se s~nadsázkou říci, že se jedná o~takové \myuv{programování}. Pro začátečníky s~{\LaTeX}em vyšla celá řada kvalitních knih. Například \cite{latex_zacatecnici} nebo v~angličtině \cite{latex_guide}. Pokud nemáte zájem si kupovat nebo půjčovat knihu, můžete využít některý z~elektronických dokumentů \cite{tex} \cite{getting}. Zda-li vás typografie opravdu zajímá, neváhejte a objednejte si některý z~časopisů, ve kterých se nachází kvalitní články z~oblasti typografie \cite{typografia} \cite{typo2}. Nebo jestli-že vás zaujal některý ze starších dílů, můžete si jej rovněž objednat \cite{typo}.
  
\section*{Pokročilá práce s~{\LaTeX}em}
Naučit se s~{\LaTeX}em není jednoduché a zabere to spoustu času, avšak pokud jste již v~používání {\LaTeX}u zběhlí, můžete v~{\LaTeX}u vypracovat třeba svoji bakalářskou nebo diplomovou práci\cite{bibtex} \cite{rozpoznavani}. Píšete-li například bakalářskou práci a nevíte jak správně citovat autory, tak můžete kouknout sem \cite{martinek}.

\section*{Závěr}
Jak jsem již zmínil. Naučit se v~{\LaTeX}u vyžaduje čas a píli, avšak odměnou za vaši vytrvalost budou dokumenty na velmi vysoké typografické úrovni a rychlost, se kterou dokážete tyto dokumenty vytvářet. Rozhodně se vyplatí si vyzkoušet pár dokumentů v~{\LaTeX}u vytvořit a třeba si tento typografický nástroj oblíbíte.

\newpage	

\bibliographystyle{czplain}
\renewcommand{\refname}{Literatura}
\bibliography{seznam}



\end{document}

